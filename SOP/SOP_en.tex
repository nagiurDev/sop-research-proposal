\documentclass[12pt,a4paper]{article}
\usepackage[utf8]{inputenc}
\usepackage{amsmath}
\usepackage{geometry}
\usepackage{CJKutf8}   
\usepackage{hyperref} 
\usepackage{fancyhdr}

\geometry{top=2.0cm,bottom=1.75cm,left=1.75cm,right=1.75cm,marginparwidth=2.00cm} 
% \usepackage[a4paper,top=2.0cm,bottom=1.75cm,left=1.75cm,right=1.75cm,marginparwidth=2.00cm]{geometry}

\newcommand{\HRule}{\rule{\linewidth}{1mm}} 

\pagestyle{fancy}
\fancyhf{} 
\fancyhead[L]{\textit{
        \begin{CJK*}{UTF8}{gbsn}RAHAMAN NAGIUR(纳吉)\end{CJK*}
} }
\fancyhead[R]{\textit{Master's admission candidate}}
\fancyfoot[C]{\thepage}

\begin{document}
\begin{CJK*}{UTF8}{gbsn}
    \thispagestyle{plain}



\begin{center}

    \Huge{ \bfseries{ Statement of Purpose (SOP)}}


    \vspace*{0.2cm}


    
\end{center}


I am Rahaman Nagiur(纳吉). My journey into the realm of computer science has been driven by a profound fascination with the intricate relationship between human communication and technology. The ability of artificial intelligence to not only understand but also replicate the nuances of human speech, particularly the emotional and expressive qualities, holds immense potential for transforming how we interact with technology and access information. This potential is particularly compelling in the context of multimedia, where realistic and engaging digital voices are essential for captivating audiences and fostering effective communication. This aspiration fuels my desire to pursue a Master's degree in Computer Science and Technology, a program renowned for its innovative research in areas such as Text-to-Speech (TTS), Intelligent Speech Analysis, Speech Enhancement, Automatic Speech Recognition (ASR), Multimedia Analysis, Acoustic Scene Analysis, and Human-Machine Interaction. Furthermore, the program's focus on advanced deep learning techniques, including Transfer Learning, Adversarial Learning, Semi-Supervised Learning, Unsupervised Learning, Active Learning, and AutoML, provides an unparalleled opportunity to develop the skills necessary to advance this rapidly evolving field. My ultimate goal is to contribute to the multimedia industry as a research scientist, specializing in creating cutting-edge multilingual TTS systems, and ideally, to establish my own startup in this field.\newline

My undergraduate studies in Computer Science and Technology at Lanzhou University of Technology provided me with a robust foundation in the theoretical and practical aspects of the discipline. Core courses in Machine Learning, Deep Learning, Natural Language Processing, Big Data Technology, Data Structures and Algorithms, SQL, and Software Engineering equipped me with the fundamental principles necessary to tackle complex computational problems. My academic performance has been consistently outstanding, culminating in two Outstanding Student Awards at Lanzhou University of Technology. These awards recognize not only my high academic standing but also my dedication to learning and my ability to apply theoretical knowledge to practical problem-solving.\newline

My undergraduate thesis delved into the intricacies of emotionally expressive Text-to-Speech synthesis. This research project required me to apply my knowledge of deep learning models, specifically transformer models, to synthesize diverse emotional qualities in speech. The project involved data collection, preprocessing, model training, and evaluation. Crucially, this thesis experience honed my research methodologies, from formulating research questions to rigorously evaluating results. My proficiency in programming languages like Python and JavaScript is further reinforced through my independent projects and collaborations with professors, bolstering my practical programming skills.\newline

% USTC's program stands out for its strong emphasis on cutting-edge research in computer science, particularly in areas relevant to my interests, such as speech processing, multimedia technologies, and AI. The research conducted at USTC, especially in the School of Computer Science and Technology, aligns perfectly with my aspirations to delve into advanced TTS and intelligent speech analysis. The program's strong ties to industry further strengthen my conviction that USTC provides an ideal environment for bridging research and practical application. The emphasis on industry-sponsored research projects will be invaluable in developing my technical skills and understanding the practical considerations of real-world applications. The robust research environment at USTC, combined with the strong support network of the faculty and fellow students, will be instrumental in achieving my ambitious research goals. The availability of resources like computing resources and access to large-scale datasets makes USTC a truly exceptional choice for my future studies. Finally, the location in China provides a unique opportunity to connect with and learn from a diverse community of researchers and potential collaborators.\newline

My long-term goal is to leverage my technical expertise to make significant contributions to the multimedia industry. I envision myself as a research scientist, actively involved in developing innovative TTS solutions that address the diverse communication needs of global audiences. The ultimate objective is the creation of a multilingual TTS startup that will empower diverse communities to connect and communicate effectively through nuanced and realistic digital voices. This master's program will be crucial in laying the foundation for this ambitious pursuit, allowing me to acquire the necessary skills and knowledge to translate research into commercially viable products. My goal is not just to build a system but to create a system that is genuinely engaging, empathetic, and relevant to the needs of the users.\newline

Beyond my academic pursuits, I have actively sought opportunities to expand my practical skills and experience. My research paper, "Fine-tuning pre-trained language models for grammatical acceptability, correction, sentiment analysis, and emotion detection," was published in the International Journal of Research in Advanced Engineering and Technology. This publication highlights my ability to conduct independent research, design experiments, analyze data, and effectively disseminate my findings to a wider academic community. My experience leading an AI study group at Lanzhou University of Technology, which involved project management, guidance of fellow students, and effective communication, has provided me with valuable team-oriented skills that are essential in a research environment. These experiences have also instilled in me the value of collaboration and the significance of addressing real-world problems through technology.\newline

My research interests encompass the development of a robust multilingual TTS framework that surpasses current limitations. I am particularly interested in specify specific research areas: 1) exploring novel ways to synthesize nuanced emotional expression in TTS; 2) developing methods for accurately capturing and reproducing language-specific prosodic features; and 3) advancing techniques for speaker attribute control to create customizable voices. These areas require a deep understanding of advanced deep learning techniques, such as transfer learning and adversarial learning, as well as advanced acoustic modeling. Furthermore, I want to explore how semi-supervised or unsupervised learning techniques can be employed to enhance the efficiency and generalizability of the TTS systems while mitigating the need for large amounts of labeled data. I aim to develop a framework that demonstrates exceptional performance across diverse languages and cultures, setting a new standard for multilingual TTS systems. \newline

After completing my Master's degree, I aspire to join a research lab or startup with a strong focus on speech technology. My aim is to leverage my knowledge and skills to further develop robust TTS models and contribute to cutting-edge research in the field. I envision actively contributing to the development of commercially viable products based on advanced TTS systems, addressing real-world problems and making a tangible impact on the accessibility and usability of multimedia content. \newline

I am confident that the Master's program in Computer Science and Technology at your University offers the ideal environment to further develop my skills and make a significant contribution to the field of speech technology. My research interests, career aspirations, and relevant experiences align seamlessly with the program's focus on cutting-edge research, advanced deep-learning techniques, and strong faculty expertise. I am eager to join a vibrant community of researchers and contribute to the ongoing advancements in TTS and intelligent speech analysis, ultimately impacting the future of human-computer interaction and multimedia accessibility. I believe my dedication, combined with the resources and guidance offered by the University, will pave the way for impactful innovation and significant progress in this field.




\end{CJK*}
\end{document}