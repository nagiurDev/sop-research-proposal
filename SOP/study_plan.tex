\documentclass[12pt,a4paper]{article}
\usepackage[utf8]{inputenc}
\usepackage{amsmath}
\usepackage{geometry}
\usepackage{CJKutf8}   
\usepackage{hyperref} 
\usepackage{fancyhdr}

\geometry{top=2.0cm,bottom=1.75cm,left=1.75cm,right=1.75cm,marginparwidth=2.00cm} 
% \usepackage[a4paper,top=2.0cm,bottom=1.75cm,left=1.75cm,right=1.75cm,marginparwidth=2.00cm]{geometry}

\newcommand{\HRule}{\rule{\linewidth}{1mm}} 

\pagestyle{fancy}
\fancyhf{} 
\fancyhead[L]{\textit{
        \begin{CJK*}{UTF8}{gbsn}RAHAMAN NAGIUR(纳吉)\end{CJK*}
} }
\fancyhead[R]{\textit{Master's admission candidate}}
\fancyfoot[C]{\thepage}

\begin{document}
\begin{CJK*}{UTF8}{gbsn}
    \thispagestyle{plain}



\begin{center}

    \large{ \bfseries{ Study Plan of RAHAMAN NAGIUR (纳吉)}}


    \vspace*{0.2cm}


    
\end{center}

My name is Rahaman Nagiur (纳吉), and I am a highly motivated and research-oriented senior undergraduate student majoring in Computer Science and Technology at Lanzhou University of Technology (LUT), where the program is taught entirely in Chinese. I am originally from Bangladesh. I am deeply fascinated by the ability of logical structures to solve complex problems, which sparked my journey into the world of computing and led me to embrace the challenge and opportunity of studying in China. This experience has not only provided me with a solid academic foundation, has also significantly cultivated my adaptability and cross-cultural understanding. \newline

Throughout my studies, I have maintained excellent academic performance (currently averaging 78/100) and have repeatedly received the “Outstanding Student Award” and the “President's Scholarship,” which fully demonstrates my dedication and effort. The core curriculum at LUT has provided me with a solid foundation in C/C++, databases, system design, machine learning (ML), natural language processing (NLP), and big data technologies. At the same time, I have proactively sought avenues to deepen my practical skills and theoretical knowledge. I have supplemented my university studies with intensive online courses, particularly completing the "CS Fundamentals With Phitron" online program (achieving a score of 92/100, ranking in the top 3\%). In this process, I honed my skills in data structures and algorithms (DSA), Python object-oriented programming (OOP), SQL, software engineering, and machine learning, completed five software engineering projects, and solved over 500 programming problems. \newline

However, my passion is truly focused on the field of artificial intelligence, especially its application in understanding and generating complex data such as human language, speech, and images. While I maintain a strong interest in intelligent speech analysis (enhancement, automatic speech recognition ASR, text-to-speech synthesis TTS), my focus has significantly expanded to the cutting-edge field of generative AI. This is fully reflected in my current undergraduate thesis project, supervised by Professor Zhao Hong, titled "Anime Character Image Generation System Based on Diffusion Models." In this ongoing project (expected to be completed in July 2025), I am delving deeply into the theory and application of diffusion models, using a U-Net architecture, Python, and frameworks like FastAPI and Next.js to build a full-stack system to generate novel anime character images based on user input. This challenging project requires me to combine deep learning theory with practical software engineering, handling everything from dataset curation and model optimization (including potential use of attention mechanisms) to backend API development and frontend user interface design. This intensive research and development experience has solidified my desire to explore more advanced generative models and their applications in depth during my graduate studies. \newline

Through coursework and self-study, the technical tools I have mastered include: proficiency in Python, JavaScript, and SQL; familiarity with C/C++, Java, and Bash; and rich experience using core machine learning/deep learning libraries such as TensorFlow, PyTorch, Pandas, and Scikit-learn, as well as tools like Node.js, Git, and Linux. I have also begun exploring specialized areas through courses like "TinyML and Efficient Deep Learning Computing" and MIT's "Introduction to Deep Learning," which demonstrates my commitment to continuous learning at the forefront of the field.  \newline

Beyond purely academic pursuits, my experience leading the Artificial Intelligence Learning Community at Lanzhou University of Technology has allowed me to share my enthusiasm, plan and host workshops on core AI concepts for a diverse group of international students, and develop valuable leadership and communication skills. My entire journey has been a process of continuous learning, adaptation, and an increasing commitment to contributing to the field of artificial intelligence, particularly in the area of generative models. I am eager to contribute the foundation I have built, the research abilities gained from published papers and thesis work, and my cross-cultural experience to the rigorous graduate studies at Yangzhou University.\newline

Complementing my strong academic foundation and research experience, I am currently engaged in a valuable internship at Peoples softtech Ltd. in Bangladesh, where I am contributing to the development of a next-generation school management system. This experience is proving particularly relevant to my graduate aspirations in generative AI, as the company is transitioning from a monolithic architecture to a microservices-based system incorporating AI-powered features. My direct involvement in designing and implementing this system, utilizing technologies like Next.js, Node.js, and Go, is providing me with practical insights into the challenges and opportunities of building scalable, data-driven applications, including those incorporating generative models. 

% ----
\section*{Why Yangzhou University and software engineering major}
I am applying to Yangzhou University for my master's degree in software engineering  because I believe it is the ideal institution to further my academic and research aspirations. The university's strong reputation in AI research, particularly in generative models, aligns perfectly with my interests and career goals. I am particularly impressed by the faculty's expertise in this area and the university's commitment to fostering interdisciplinary collaboration.\newline

I am eager to engage with faculty members who are at the forefront of AI research and contribute to projects that have a meaningful impact on society. I am particularly drawn to Yangzhou University due to its strong emphasis on research and innovation in artificial intelligence. The university's commitment to fostering interdisciplinary collaboration aligns perfectly with my aspirations to explore the intersection of technology and creativity. I am excited about the opportunity to learn from leading experts in the field and contribute to the advancement of generative AI technologies.\newline

I am confident that my dedication, adaptability, and enthusiasm for learning will enable me to thrive in the rigorous academic environment at Yangzhou University. I look forward to the opportunity to contribute to the university's research initiatives and collaborate with faculty and fellow students on innovative projects that push the boundaries of artificial intelligence.\newline

\section*{My study plan at Yangzhou University}
My study plan at Yangzhou University is to focus on advanced coursework in artificial intelligence, particularly in the areas of generative models and their applications. I plan to take courses that deepen my understanding of machine learning, deep learning, and natural language processing, while also exploring the latest developments in generative AI technologies.
I am particularly interested in courses that cover advanced topics in generative models, including diffusion models, GANs, and VAEs. I also plan to engage in hands-on projects that allow me to apply theoretical concepts to real-world problems, further enhancing my practical skills and research capabilities.

\subsection*{First Academic Year: Deepening Knowledge}
In my first year, I plan to immerse myself in advanced coursework and research opportunities. I will focus on deepening my understanding of machine learning, deep learning, and natural language processing, while also exploring the latest developments in generative AI technologies. I aim to take courses that cover advanced topics in generative models, including diffusion models, GANs, and VAEs. Additionally, I will actively seek out research opportunities within the university's AI research labs, collaborating with faculty members and fellow students on cutting-edge projects.

\subsection*{Second Academic Year: Research and Thesis Development }
In my second year, I plan to focus on my thesis research, building upon the foundation established in my first year. I will work closely with my advisor to refine my research questions and methodologies, ensuring that my thesis contributes meaningfully to the field of generative AI. I aim to publish my research findings in reputable conferences or journals, further solidifying my academic credentials and contributing to the advancement of knowledge in this area.\newline

\subsection*{Long-Term Goals: Contributing to the Field of Generative AI }
My long-term goal is to become a leading researcher or engineer specializing in generative AI. I aspire to contribute to the development of next-generation generative models for images, speech, and other modalities. I am particularly interested in exploring the applications of generative AI in various domains, including gaming, media, and design. I believe that my background in software engineering and my experience in developing generative systems will enable me to contribute meaningfully to ongoing research projects at Yangzhou University. I am also eager to collaborate with fellow students and faculty members to explore new ideas and push the boundaries of what is possible with generative AI.\newline

By committing to executing this plan, leveraging the excellent resources and faculty of Yangzhou University, and actively integrating into the vibrant academic and technical environment, I am confident that I can make important contributions and achieve my goal of becoming a highly skilled researcher and engineer in the field of generative artificial intelligence and its applications.\newline


\section*{Research background and Academic achievements}

My academic journey has been marked by a strong research drive and a pursuit of excellence, as evidenced by my official recognitions, tangible research outcomes, and numerous project practices, culminating in my current undergraduate thesis project.

\subsection*{Research Assistant, School of Materials Science and Engineering, LUT}
Applied machine learning techniques (data preprocessing, feature engineering, model training, and validation) to optimize the selection of biocompatible polymers. This interdisciplinary role honed my core machine learning skills and demonstrated my adaptability in applying technology to entirely new domains.


\subsection*{Research Contributions (Publications):}
During my undergraduate studies, I have published two research papers as the first author:
\begin{itemize}
    \item Rahaman Nagiur, Al-Muqaddam Anas, et al. (2024), "Fine-tuning pre-trained language models for grammatical acceptability, correction, sentiment analysis, and emotion detection." International Journal of Research in Advanced Engineering and Technology, 10(2), 42-49. 
    \item Rahaman Nagiur, Perfilev Dmitrii (2024), "Navigating the DevOps landscape: Insights and perspectives." International Journal of Research in Advanced Engineering and Technology, 10(1), 27-29. 
\end{itemize}


\subsection*{Academic Achievements and Awards}
\begin{itemize}
    \item \textbf{Outstanding Student Award} Recipient, Lanzhou University of Technology (2023): Recognizing consistent academic excellence and positive contributions.
    \item \textbf{Outstanding Student Award} Recipient, Lanzhou University of Technology (2022): Recognizing consistent academic excellence and positive contributions.
    \item \textbf{Presidential Scholarship} Recipient, Lanzhou University of Technology (2021): Demonstrating outstanding academic potential recognized upon enrollment.
\end{itemize}

\subsection*{Undergraduate Thesis Project (In Progress)}
My current undergraduate thesis project (Lanzhou University of Technology, supervised by Professor Zhao Hong, expected completion in July 2025), titled "Anime Character Image Generation System Based on Diffusion Models," aims to leverage artificial intelligence to efficiently generate diverse anime characters. The project uses a diffusion model based on U-Net using PyTorch, and I am developing a full-stack system with a backend in Python, FastAPI and a frontend in Next.js to enable generation based on user-specified conditions. As the sole student researcher, I am responsible for the entire project lifecycle, from dataset curation and model training to full-stack implementation and testing. This project has provided me with critical practical experience in generative AI, deep learning development, full-stack software engineering, and the entire research-to-deployment process, preparing me for graduate studies.These accomplishments, combined with my research contributions, thesis work, and various project practices, reflect my strong research capabilities, pursuit of academic excellence, and proactive attitude in applying and expanding my knowledge of computer science, software engineering, and artificial intelligence.

% ----
\section*{The social activities, internship and voluntary work}
While my primary focus has consistently been on academic and research excellence, I am also deeply aware of the importance of collaboration, leadership, and community engagement. I have actively cultivated these abilities through participation in student organizations at Lanzhou University of Technology.


\subsection*{Software Engineer Intern, Peoples softtech Ltd., Bangladesh }
As a member of the development team, I am directly involved in the design and implementation of the new system, contributing to both the frontend and backend components. This includes working with the modern technology stack, which comprises a Next.js (React.js) frontend, a Node.js/Go backend, and a database designed to support AI integration.

\subsection*{Head of the Artificial Intelligence Learning Community, LUT}
This was a significant leadership practice. I proactively initiated and led a weekly AI learning community, specifically geared towards the international student population, encompassing students at the doctoral, master's, and undergraduate levels.My responsibilities were quite comprehensive, involving the planning and design of a 10-weekend series of workshops covering core AI concepts (from foundational knowledge to introductory material on deep learning, natural language processing, and computer vision), creating presentation materials, teaching the workshop content in an engaging manner, promoting the activities, and fostering a collaborative learning environment for participants with diverse backgrounds.This role required significant organizational, planning, cross-cultural communication, and teaching skills. I learned to tailor complex technical information to different backgrounds and facilitate active participation among members. This greatly enhanced my leadership, public speaking, and teaching abilities.


\subsection*{Teamwork Activities}
I participated in numerous course projects (including 5 software engineering projects completed through Phitron.io, as well as complex group projects within the Lanzhou University of Technology curriculum), which required substantial teamwork, collaborative coding using Git, joint problem-solving, and completing tasks within deadlines.My undergraduate graduation design project, while primarily an individual research endeavor, also required adhering to strict software engineering practices, including planning, modular design, and potential future collaboration should the project expand. The process of developing a full-stack application (Next.js frontend, Python, FastAPI backend) demonstrated my ability to manage complex, multi-component technical projects.As an active member of the international student community at Lanzhou University of Technology, I frequently interacted and collaborated with students from diverse backgrounds on both academic and social levels, further enhancing my cross-cultural communication and teamwork skills.


\section*{Additional Supplementary Information}
Beyond the specific details already outlined above, I would like to highlight a few additional points to further clarify my application background and aspirations.

\subsection*{Long-Term Career Plan}
Pursuing a master's degree at Yangzhou University is a crucial step toward achieving my long-term goal of becoming a leading researcher or engineer specializing in generative AI. After graduation, I hope to join an innovative research lab or a forward-thinking technology company to contribute to the development of next-generation generative models for images, speech, or other modalities. My current graduation design work, which focuses on developing diffusion model-based generative systems, strongly reaffirms my passion for this field and strengthens my career aspirations to pursue cutting-edge generative technologies. In the longer term, I aspire to lead research teams, make significant contributions to open-source AI projects (such as those surrounding diffusion models), or apply these technologies to address social challenges in areas such as the creative industries, personalized education, or accessibility services.

\subsection*{Motivation for Choosing to Study in China and Yangzhou }
Given my successful completion of a challenging undergraduate program taught entirely in Chinese at Lanzhou University of Technology, I am eager to continue my studies in China, particularly choosing Yangzhou, a well-considered decision. As a global center for AI research and industry, Yangzhou provides an unparalleled environment for accessing cutting-edge developments, connecting with leading professionals, and understanding AI applications in a highly active market, especially generative AI applications concentrated in industries like gaming, media, and design. 

\subsection*{Hobbies and Interests}
I have a strong interest in reading, particularly in the fields of computer science and artificial intelligence. I also enjoy engaging with the latest developments in generative AI, including diffusion models and their applications. In addition to my academic pursuits, I am an avid fan of anime and manga, which has inspired my current undergraduate thesis project. I also enjoy playing video games, which has further fueled my interest in the intersection of technology and creativity.

\subsection*{Interdisciplinary Perspective}
I believe that the future of artificial intelligence lies in its ability to integrate and collaborate with various fields, including computer science, psychology, linguistics, and art. My interdisciplinary background and experiences have equipped me with a unique perspective on how generative AI can be applied to diverse domains, from enhancing user experiences in gaming to improving accessibility in education. I am excited about the potential of generative AI to revolutionize industries and create innovative solutions to complex problems.

\subsection*{Conclusion}
In conclusion, I am deeply committed to pursuing a master's degree in artificial intelligence at Yangzhou University, where I can further develop my skills and knowledge in generative AI. I am eager to contribute to the vibrant academic community at Yangzhou University and collaborate with faculty and fellow students on cutting-edge research projects. I believe that my strong academic background, research experience, and passion for artificial intelligence make me a suitable candidate for this program. I am excited about the opportunity to learn from leading experts in the field and contribute to the advancement of generative AI technologies.


% move to top




\end{CJK*}
\end{document}