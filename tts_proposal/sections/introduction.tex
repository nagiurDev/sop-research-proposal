\section*{Introduction}

The rapid advancement of artificial intelligence (AI) has spurred significant growth in speech technologies, impacting sectors from entertainment to accessibility. Demand for natural and emotionally expressive text-to-speech (TTS) systems is surging, driven by applications like audiobooks, audio dramas, and interactive experiences. Current state-of-the-art neural TTS models, while effective in English, face substantial challenges when applied to other languages, especially low-resource languages like Bengali and Chinese. These challenges stem from the complexities of language-specific prosody (intonation, rhythm, stress) and tonal variations. Consequently, synthesized speech often lacks the emotional expressiveness and nuanced prosodic features of human voices, resulting in a robotic quality. \newline

Despite generating understandable speech, existing TTS models often struggle to capture the full range of human emotion and prosodic subtleties. This limitation is particularly acute in low-resource languages due to limited training data and the unique characteristics of prosodic patterns. Furthermore, controlling attributes like age, gender, and emotional expression within synthesized speech remains a significant hurdle. This restricts the practical application of TTS beyond basic transcription tasks. \newline

This research aims to develop an industry-ready multilingual TTS framework that addresses these limitations by generating human-like voice actors for English, Bengali, and Chinese. The framework will focus on creating high-quality synthesized speech with controllable attributes including age, gender, and emotional expression. A critical component is the development of robust models capable of handling the specific prosodic and tonal characteristics of each language, particularly the unique features of Bengali and Chinese. This includes addressing data scarcity for Bengali and refining prosody modeling techniques for all three languages. The resulting system will improve the quality and naturalness of synthesized speech, enabling more engaging and convincing content across a wider range of applications. \newline

Beyond enhancing storytelling and entertainment, this framework has applications in assistive technologies, virtual assistants, AI-powered customer service, and educational tools. The ability to tailor voices to various emotional states and speaker characteristics will enable immersive and engaging interactive experiences across diverse linguistic and cultural contexts, pushing the boundaries of what's possible in generative speech applications.
