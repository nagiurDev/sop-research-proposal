\section*{Problem Statement}

Existing text-to-speech (TTS) systems, while showing improvements in naturalness, fall short of producing truly human-like voices, especially in multilingual contexts and for under-resourced languages like Bengali and Chinese. These systems often lack nuanced emotional expression, failing to accurately convey a range of emotions (joy, sadness, anger) and adapt their prosody (intonation, rhythm, stress) to the subtleties of the spoken text. This deficiency is particularly pronounced in multilingual applications and for under-resourced languages, where limited annotated data and the need to capture culturally specific prosodic nuances greatly complicate the task. Consequently, the synthesized voices lack the authenticity and diverse characteristics of human voice actors, significantly limiting their applicability in entertainment, education, and accessibility. This research aims to address these limitations by developing an industry-ready multilingual TTS framework capable of synthesizing human-like voices for English, Bengali, and Chinese. This framework will incorporate advanced emotional expression techniques and model language-specific prosodic variations to produce more expressive and engaging digital voice actors.