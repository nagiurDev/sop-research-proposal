\section*{研究目标}

\begin{enumerate}
    \parsep=20pt
    \item \textbf{开发一个强大的多语言文本转语音 (TTS) 框架:}  该框架将支持为英语、孟加拉语和中文合成高质量语音。
    \item \textbf{建模和合成细致的情感表达:} 该框架必须能够准确地传达合成语音中的各种情感(例如快乐、悲伤、愤怒)。这包括开发在 TTS 过程中准确编码和解码情感的技术。
    \item \textbf{准确捕捉和再现语言特定的韵律特征:} 该框架应该有效地建模和再现英语、孟加拉语和中文的独特韵律特征,包括语调、节奏、重音模式和声调变化。这需要针对每种语言的特定建模技术。
    \item \textbf{实现说话人属性的控制:} 该框架应允许在合成过程中控制各种说话人属性(例如年龄、性别、声线类型)。这增强了生成语音的灵活性和多功能性。
    \item \textbf{展示框架的产业适用性:} 开发的框架必须展示其在实际应用(例如有声读物、有声剧和纪录片)中的实用性和潜力。这可能包括展示系统在生产环境中的速度和效率。
\end{enumerate}