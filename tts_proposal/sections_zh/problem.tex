\section*{问题陈述}

现有的文本转语音 (TTS) 系统,虽然在自然度方面有所改进,但仍难以生成真正逼真的语音,尤其是在多语言环境和资源匮乏的语言(如孟加拉语和中文)中。这些系统往往缺乏细致的情感表达,无法准确传达各种情绪(快乐、悲伤、愤怒),并且无法根据语音文本的细微差别调整其韵律(语调、节奏、重音)。这种不足在多语言应用和资源匮乏的语言中尤为明显,因为有限的标注数据和捕捉特定文化韵律细微差别的需求极大地增加了任务的复杂性。因此,合成的语音缺乏真实性和人类配音演员的多样化特征,极大地限制了它们在娱乐、教育和无障碍领域的应用。本研究旨在通过开发一个可投入产业应用的多语言 TTS 框架来解决这些限制,该框架能够为英语、孟加拉语和中文合成逼真的人声。该框架将整合先进的情感表达技术,并对语言特定的韵律变化进行建模,从而产生更具表现力和吸引力的数字配音演员。