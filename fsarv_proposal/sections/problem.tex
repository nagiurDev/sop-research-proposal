\section*{Problem Statement}

The development of robust action recognition systems is significantly hampered by the need for extensive labeled data, a resource often scarce for novel or specialized actions. This data scarcity is a major obstacle for few-shot action recognition, where models must learn to identify new actions from only a handful of examples. Traditional deep learning models, trained on massive datasets, struggle to generalize effectively in these scenarios, frequently overfitting the limited training data, resulting in poor performance on unseen instances. Existing approaches, such as meta-learning and transfer learning, offer partial solutions but often face limitations. Data augmentation, a common technique for artificially increasing the training data size, typically relies on random transformations that may not generate semantically meaningful variations of the action, thereby limiting its effectiveness in few-shot learning, where understanding subtle nuances is crucial. Therefore, there is a need for a more efficient and semantically-aware approach to data augmentation.